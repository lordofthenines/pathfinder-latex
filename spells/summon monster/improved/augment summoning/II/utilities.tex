\usepackage{color}
\usepackage{fp}
\usepackage{ifthen}

% -- New Colors --
\definecolor{acidgreen}{RGB}{155,155,0}
\definecolor{blue}{RGB}{0,0,255}
\definecolor{coldblue}{RGB}{51, 153, 255}
\definecolor{electricblue}{RGB}{55, 155, 155}
\definecolor{darkgreen}{RGB}{0,100,0}
\definecolor{firered}{RGB}{255,100,0}
\definecolor{lightyellow}{RGB}{225,200,0}
\definecolor{shadecolor}{RGB}{0,25,0}

% -- New Commands ---

% This command calculates the final ability score #1 calculated from the sum of the provided
% modifiers.
%
% #1 - The name of the ability score that will receive the final value.
% #2 - Base Score
% #3 - Racial Modifier
% #4 - Level Up Bonus
% #5 - Inherent Bonus
% #6 - Enhancement Bonus
% #7 - Misc. Bonus
\newcommand{\ability}[7]{\FPeval{#1}{clip(#2+(#3)+#4+#5+#6+(#7))}}

% This command calculates the ability modifier of #2 and places that value in the variable
% named #1.
%
% #1 - The name of the variable that will receive the final value.
% #2 - Base Score
\newcommand{\abilitymodifier}[2]{\FPeval{\temp}{clip(#2 - 10)}%
							  \FPifneg{\temp}%
								 \FPeval{#1}{round((#2 - 10)/2, 0)}%
							  \else%
                                     \FPeval{#1}{trunc((#2 - 10)/2, 0)}%
							  \fi}

% This macro changes the color of the displayed text to a cool green to denote acid damage.
\newcommand{\aciddamage}[1]{\color{acidgreen}\textbf{#1}\color{black}}
							
% This command calculates a character's armor class #1 by adding together #2-#6. 
%
% #1 - The name of the constant that will receive the final value.
% #2 - Armor Bonus
% #3 - Shield Bonus
% #4 - Natural Armor Bonus
% #5 - Dexterity Modifier
% #6 - Deflection Bonus
% #7 - Size Modifier
\newcommand{\armorclass}[7]{\FPeval{#1}{clip(10+#2+#3+#4+(#5)+#6+(#7))}}

% This command calculates the attack bonus for a particular attack denoted by #1 by adding
% together the modifiers given by #2-#6. 
%
% #1 - The name of the attack that will receive the final value.
% #2 - Base Attack Bonus
% #3 - Ability Modifier
% #4 - Enhancement Bonus
% #5 - Size Modifier
% #6 - Feat Bonus
\newcommand{\attack}[6]{\FPeval{#1}{clip(#2+(#3)+#4+(#5)+(#6))}}

% This macro changes the color of the displayed text to a cool blue to denote cold damage.
\newcommand{\colddamage}[1]{\color{coldblue}\textbf{#1}\color{black}}

% This command calculates the final Combat Maneuver Bonus (CMB) #1 calculated from the sum
% of #2-#4.
%						 
% #1 - The name of the constant that this value will be placed in
% #2 - Base Attack Bonus
% #3 - Dexterity Modifier (w/o Agile Maneuvers feat this is Strength Modifier)
% #4 - Size Modifier
\newcommand{\combatmaneuverbonus}[4]{\FPeval{#1}{clip(#2+(#3)+(#4))}}

% This command calculates the final Combat Maneuver Defense (CMD) #1 calculated from the sum
% of #2-#5.
%
% #1 - The name of the constant that this value will be placed in
% #2 - Base Attack Bonus
% #3 - Strength Modifier
% #4 - Dexterity Modifier
% #5 - Size Modifier
\newcommand{\combatmaneuverdefense}[5]{\FPeval{#1}{clip(10+#2+(#3)+(#4)+(#5))}}

% This macro changes the color of the displayed text to a cool red to denote fire damage.
\newcommand{\electricitydamage}[1]{\color{electricblue}\textbf{#1}\color{black}}

% This macro changes the color of the displayed text to a pleasing green to denote magical 
% enhancement.
\newcommand{\enhancedstat}[1]{\color{darkgreen}\textbf{#1}\color{black}}

% This macro changes the color of the displayed text to a cool red to denote fire damage.
\newcommand{\firedamage}[1]{\color{firered}\textbf{#1}\color{black}}

% This macro changes the color of the displayed text to an awful red to denote a penalty.
\newcommand{\reducedstat}[1]{\color{red}\textbf{#1}\color{black}}

% This command calculates a saving throw value from the sum of #2-#6 and places the result
% in the variable named #1.
%
% #1 - The name of the saving throw that will receive the final value.
% #2 - Base Save
% #3 - Ability Modifier
% #4 - Feat Bonus
% #5 - Resistance Bonus
% #6 - Miscellaneous Bonus
\newcommand{\savingthrow}[6]{\FPeval{#1}{clip(#2+(#3)+#4+#5+#6)}}

% This command calculates a skill value from the sum of #2-#9 and places the result in the
% the variable named #1.
%
% #1 - The name of the skill that will receive the final value.
% #2 - Ranks
% #3 - Ability Modifier
% #4 - Class Skill Bonus
% #5 - Racial Modifier
% #6 - Feat Bonus
% #7 - Enhancement Bonus
% #8 - Misc. Modifier
% #9 - Armor Check Penalty
\newcommand{\skill}[9]{\FPeval{#1}{clip(#2+(#3)+#4+(#5)+#6+#7+(#8)+(#9))}}
